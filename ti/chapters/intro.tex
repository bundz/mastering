\chapter{Introdução}

A Computação em Nuvem, o sonho antigo da computação como commodity, tem o potencial de transformar grande parte de indústria da tecnologia da informação. Fazendo com que os softwares sejam ainda mais atrativos como serviço e transformando a maneira como os hardwares são projetados e comprados~\cite{Armbrust:2009}. Com a Computação em Nuvem desenvolvedores com idéias inovadoras não precisam mais de grande capital inicial para lançar suas idéias. Além disso, empresas com tarefas extremamente grandes podem coseguir seus resultados tão rápido quanto seus programas conseguem escalar, afinal usar 1000 servidores por uma custa o mesmo que 1 servidor por 1000 horas. Essa elasticidade de recursos, sem custo adicional, é revolucionária na história da tecnologia da informação~\cite{Armbrust:2009}.

A Computação em Nuvem refere-se tanto a aplicações entregues como serviços pela internet quanto a hardware e sistemas que fornecem esses serviços. Os serviços veem sendo chamados de \textit{Software-as-a-Service} (SaaS). O hardware e softare do datacenter é o que é chamado de Nuvem. Quando uma Nuvem está disponível para o público geral, chamamos de Nuvem Pública e o serviço vendido de Utility Computing. Se uso o termo Nuvem Privada para referenciar a datacenters internos de empresas ou outras organizações não disponíveis para o público geral. 

Além do modelo SaaS existe os modelos \textit{Plataform-as-a-service} (PaaS) que visa oferecer plaformas de desenvolvimento como serviço e \textit{Infraestructure-as-a-Service} (IaaS) que visa oferecer infraestrutura como serviço. Diversas soluções IaaS veem sendo desenvolvidas, muitas delas são plataformas de código aberto como: OpenStack, Nimbus, Eucalyptus e OpenNebula.

O objetivo deste trabalho é fazer um estudo sobre Computação em Nuvem e um estudo sobre Plataformas da Nuvem de Código Aberto. 


Este trabalho esta dividido em 4 capítulos. O primeiro capítulo é a Introdução. No segundo capítulo é apresentado um estudo sobre Computação em Nuvem, seu histórico, definições, características, modelos de negócio, arquitetura, tipos de nuvem e desafios. No terceiro capítulo é apresentado um estudo sobre as plataformas da Nuvem de Código Aberto, onde é feito um estudo mais aprofundado sobre o software OpenStack.
